Виртуальный музей – это информационная система, содержащая концептуально единую электронную коллекцию или совокупность коллекций экспонатов, имеющая характеристики музея и позволяющая осуществлять научную, просветительскую, экспозиционную и экскурсионную деятельность в виртуальном пространстве.

Игра \char`\"{}\+Virtual museum\char`\"{} является примером виртуального музея. Сейчас в игре три зала с различной тематикой и наполнением.

Данный проект реализован в качестве курсового проекта курса \char`\"{}АРХИТЕКТУРНОЕ ПРОЕКТИРОВАНИЕ И ПАТТЕРНЫ ПРОГРАММИРОВАНИЯ\char`\"{}.

Игра написана на Unity с использованием множества признанных паттернов программирования. 